\documentclass[10pt]{beamer}

\usepackage{enumitem}
\usepackage{graphicx}
\usepackage{amsmath,amssymb}
\usepackage{physics}
\usepackage{animate}
\usepackage[utf8]{inputenc}
\usepackage{bm}
\newcommand{\bmva}[1]{\vec{\bm#1}}

\usetheme{Berlin}

\title[Optimisation and Backflow]{Wavefunction Optimisation and Backflow in Quantum Monte Carlo Simulations:}
\subtitle{}
\author[Clio Johnson]{Clio~Kennedy~Johnson}
%\institute[FST]{Faculty of Science and Technology}
\date[2023-10-17]{17th of October, 2023}
\titlegraphic{%
    \includegraphics[scale=.29]{./images/heatmap2.png}
}

\begin{document}

\begin{frame}
    \titlepage%
\end{frame}


\begin{frame}
    \frametitle{Introduction \& Overview}
    \begin{itemize}
        \item[\textbullet] Introduction to quantum Monte Carlo (QMC) methods, particularly variational Monte Carlo (VMC).
        \item[\textbullet] A look at wavefunction types and optimisation
        \item[\textbullet] A brief exploration of backflow displacements
    \end{itemize}
\end{frame}


\begin{frame}[allowframebreaks]
    \frametitle{Quantum Monte Carlo (QMC)}
    \textbf{First of all, what is a Monte Carlo method?}\medskip\newline
    Monte Carlo are numerical methods that obtain a result through random sampling. They involve an overall picture generated from a large number ($M$) of iterations.\medskip\newline
    They are useful for systems with many, \textit{many} degrees of freedom as the error in a result scales $\propto\frac{1}{\sqrt{M}}$ irrespective of the dimensionality of a system.\medskip\newline % Mention that this is a consequence of the Central Limit Theorem.
    You can do Monte Carlo integration on integrals that looks like this:
    \begin{equation}
        E = \int\dd{\vb{x}}\rho\pqty{\vb{x}}f\pqty{\vb{x}}
        \approx\sum_{n=1}^{M}\frac{f\pqty{\vb{x}_n}}{M}
    \end{equation}
    \framebreak%

    \textbf{A simple Monte Carlo Example:}\medskip\newline
    Compute $\pi$ from evaluating a circular integral:
    \begin{figure}
        \centering
        % Put a graphic of quarter circle MC integration here
        \animategraphics[controls,scale=0.25]{3}{./figs/pi_gif/pi_30K-}{0}{9}
        \caption{%
            By nicoguaro - Own work, CC BY 3.0,
            \url{https://commons.wikimedia.org/w/index.php?curid=14609430}
        }
    \end{figure}
    \framebreak%

    \textbf{What are QMC methods?}\medskip\newline
    QMC methods aim to solve many-body Schr\"odinger equations iteratively. They are used to estimate ground state energies of these many-body systems as well as optimise wavefunctions.\medskip\newline
    \begin{equation}
        \hat{H}=\sum\limits_{i}^{N_e}\frac{\nabla_i^2}{2}+\sum\limits_{i}^{N_e}\sum\limits_{j\neq i}^{N_e}\frac{1}{\vqty{\va{r}_{ij}}}-\sum\limits_{i}^{N_e}\sum\limits_{a}^{N_I}\frac{Z_a}{\vqty{\va{r}_{ia}}}+\sum\limits_a^{N_I}\sum\limits_b^{N_I}V_{ab}
    \end{equation}
    \framebreak%

    \textbf{Types of QMC}\medskip\newline
    There are two QMC algorithms that are most commonly used:
    \begin{itemize}
        \item[\textbullet] \textbf{Variational quantum Monte Carlo (VMC)}\newline
        Takes a trial wavefunction with electron positions, proposes a move in configuration space and then either accepts or rejects it according to the Metropolis-Hastings method.\newline
%       Other free parameters in the wavefunction can be optimised throughout the simulation. The simulation produces a variational estimate of the ground state energy through a rolling mean of local energies calculated at each iteration. The accuracy of this energy is dependent on a number of factors, including how well optimised the wavefunction parameters are.\newline
        \item[\textbullet] \textbf{Diffusion quantum Monte Carlo (DMC)}\newline
        DMC solves an imaginary time Schr\"odinger equation. A large number of configurations are diffused throughout the simulation, and their density is used to recover the ground state wavefunction. Most DMC simulations are \textbf{Fixed Node} DMC
    \end{itemize}
    \framebreak%

    \textbf{VMC Formulation}\medskip\newline
    Given a configuration of $N$ electrons denoted $\vb{R}=\pqty{\vb{r}_1,\vb{r}_2,\ldots,\vb{r}_N}$, with an $N$ body Hamiltonian $\hat{H}$.\newline
    If we provide an approximate trial wavefunction $\psi_T\pqty{\vb{R}}$. The variational principle ensures:
    \begin{equation}
        \int\dd{\vb{R}}\psi_T^*\pqty{\vb{R}}\hat{H}\psi_T\pqty{\vb{R}}\geq E_0
    \end{equation}
    So we're free to see the configuration $\vb{R}$ as a set of variational parameters, used to home in on a variational upper bound to the ground state energy
    \framebreak%

    \textbf{VMC Formulation Continued}\medskip\newline
    A Markov chain Monte Carlo method called the Metropolis algorithm is used to sample a series of points from the probability density function $\vqty{\psi_T\pqty{\vb{R}}}^2$.\medskip\newline
    \begin{equation}
        \int\dd{\vb{R}}
        \vqty{\psi_T\pqty{\vb{R}}}^2
        E_L\pqty{\vb{R}}
        \geq E_0,
    \end{equation}
    where
    \begin{equation}
        E_L\pqty{\vb{R}}=\frac{\hat{H}\psi_T\pqty{\vb{R}}}{\psi_T\pqty{\vb{R}}}.
    \end{equation}
    \framebreak%

    \begin{figure}
        \centering
        \includegraphics[scale=0.37]{./images/control_process_diamond_ae2.png}
        \caption{Recovered autocorrelation function from the mean of 40,000
        VMC simulations on 8 cell all-electron diamond.}
    \end{figure}
    \framebreak%

    At each timestep a new configuration is proposed, usually from a Gaussian distribution:
    \begin{equation}
        \va{R}'\sim N\pqty*{\va{R}_m,\sigma^2},
    \end{equation}
    which has the conditional probability $P\pqty*{\va{R}'|\va{R}_m}$. It is then accepted according to
    \begin{equation}
        a_m=\min\pqty{\frac{\vqty*{\psi_T\pqty*{\va{R}'}}^2P\pqty*{\va{R}_m|\va{R}'}}{\vqty*{\psi_T\pqty*{\va{R}_m}}^2P\pqty*{\va{R}'|\va{R}_m}},1}.
    \end{equation}
    \framebreak%

    \begin{itemize}
        \item[\textbullet] \textbf{Diffusion quantum Monte Carlo (DMC)}\newline
        DMC propagates walkers through imaginary time, using a birth-death algorithm dictated by the trial wavefunction $\psi_T$, and evaluating the ground state energy based on regions of walker density.
    \end{itemize}
    \framebreak%

    \textbf{Fixed Nodes!}
    \begin{itemize}
        \item[\textbullet] DMC uses regions of positive walker density.
        \item[\textbullet] Nodes are points at which $\psi_T\pqty{\vb{R}}=0$.
        \item[\textbullet] Remember our system is $3N_e$ Dimensional.
        \item[\textbullet] Nodal structure of the wavefunction is a $3N_e-1$
        dimensional hypersurface.
    \end{itemize}
    \framebreak%

    \begin{figure}
        \includegraphics[scale=0.22]{./images/nodal-surface-slice.png}
        \caption{%
            A ``slice'' of the nodal surface of a 161 electron wavefunction,
            courtesy of Foulkes \textit{et al.}\cite{Foulkes2001} and Ceperley \textit{et al.}.
        }
    \end{figure}
    \framebreak%

    \begin{itemize}
        \item[\textbullet] We sometimes need to alter the nodal surface of our trial wavefunction!
        \item[\textbullet] This can be done through VMC wavefunction optimisation by implementing \textbf{backflow} (more on that later).
    \end{itemize}

\end{frame}

\begin{frame}[allowframebreaks]
    \frametitle{Wavefunction optimisation}
    Starting with the simple $N$-body antisymmetric noninteracting wavefunction, the Slater Determinant:
    \begin{equation}
        S\pqty*{\va{R}}=\frac{1}{\sqrt{N_e!}}\text{det}\vmqty{%
            \chi_0\pqty*{\va{r}_0}     & \chi_1\pqty*{\va{r}_0}     & \cdots & \chi_{N_e}\pqty*{\va{r}_0} \\
            \chi_0\pqty*{\va{r}_1}     & \chi_1\pqty*{\va{r}_1}     & \cdots & \chi_{N_e}\pqty*{\va{r}_1} \\
            \vdots                     & \vdots                     & \ddots & \vdots                     \\
            \chi_0\pqty*{\va{r}_{N_e}} & \chi_1\pqty*{\va{r}_{N_e}} & \cdots & \chi_{N_e}\pqty*{\va{r}_{N_e}}
        }.
    \end{equation}
    \framebreak%

    Obviously there are electron correlation effects unnacounted for using a singular Slater Determinant. It could be possible to ``tailor'' a wavefunction to a particular system by using a multideterminant wavefunction, as is commonly done with post-Hartree-Fock methods. This comes with its unique benefits and drawbacks but most notably results in a loss of generality.\medskip\newline
    The most well used technique in VMC, is to use a \textbf{Jastrow correlation factor} to recover a good chunk of the correlation energy.
    \framebreak%

    \textbf{The Slater-Jastrow wavefunction:}
    \begin{equation}
        \psi_T\pqty*{\va{R}}=S\pqty*{\va{R}}\exp\bqty*{J\pqty*{\va{R}}}
    \end{equation}
    where the Jastrow factor (in CASINO) is chosen to be homogeneous and isotropic. It is symmetric under the exchange of electrons as symmetry is handled by the Slater determinant, and it is chosen such that the kinetic energy is well defined.
    \framebreak%

    \textbf{An example Jastrow term:}\medskip\newline
    The Jastrow factor contains a number of terms that reflect the physics of the system, a simple example would be the $u$ term:
    \begin{equation}
        J\pqty*{\va{R}}=\sum_{i=1}^{N_e-1}\sum_{j=i+1}^{N_e}u\pqty*{\va{r}_{ij}}+\ldots
    \end{equation}
    This term, and others, are outlined by Drummond \textit{et al.}\cite{Drummond2004}.\medskip\newline
    The form of $u\pqty{\va{r}_{ij}}$ is a polynomial plus a constant term with a decay function, where the coefficients, denoted $\alpha_l$, are used as optimisable parameters.
    \framebreak%

    The Slater-Jastrow wavefunction has its weaknesses, that being that it cannot modify the Nodal surface, since $\exp\bqty*{J}>0$.
\end{frame}

\begin{frame}[allowframebreaks]
    \frametitle{Slater-Jastrow-backflow Wavefunctions}
    \textbf{Backflow Displacement}
    \begin{itemize}
        \item[\textbullet] Nodal surface is determined by $S\pqty*{\va{R}}$.
        \item[\textbullet] We can simulate interaction by replacing $\vb{R}$ with a configuration of pseudo-coordinates:
        \begin{equation}
            \va{X}\pqty*{\va{R}}=\va{R}+\bmva{\xi}\pqty*{\va{R}}
        \end{equation}
        \item[\textbullet] We give $\bmva{\xi}\pqty*{\va{R}}$ an appropriate form dependent on several variational parameters. Allowing us to optimise the nodal surface variationally\cite{LopezRios2006}.
        \item[\textbullet] Leading to a wavefunction of the following form:
        \begin{equation}
            \psi_T=S\pqty*{\va{X}\pqty*{\va{R}}}\exp\pqty*{J\pqty*{\va{R}}}
        \end{equation}
    \end{itemize}
    \framebreak%

    \textbf{How to calculate Backflow displacement}
    \begin{itemize}
        \item[\textbullet] Our approximation is comprised of two-body, electron-nucleus, and electron-electron-nucleus terms.
        \item[\textbullet] I am currently implementing a three-body backflow contribution (electron-electron-electron interactions).
    \end{itemize}
    \framebreak%

    \begin{align}
        \bmva{\xi}_i\pqty*{\va{R}}=&
        \bmva{\xi}_i^{\text{e-e}}\pqty*{\va{R}}+
        \bmva{\xi}_i^{\text{e-n}}\pqty*{\va{R}}+
        \bmva{\xi}_i^{\text{e-e-n}}\pqty*{\va{R}}+
        \bmva{\xi}_i^{\text{e-e-e}}\pqty*{\va{R}}
        \\
        \bmva{\xi}_i^{\text{e-e-e}}\pqty*{\va{R}}=&
        \sum\limits_{\substack{j\neq i\\j=1}}^{N_e}
        \sum\limits_{\substack{k\neq i\\k>j\\k=1}}^{N_e}
        \omega_A\pqty*{r_{j'k'},r_{i'k'},r_{i'j'}}
        \pqty*{\va{r}_{ik'}\delta_{ij'}+\va{r}_{ij'}\delta_{ik'}}+\nonumber\\
        &\omega_B\pqty*{r_{j'k'},r_{i'k'},r_{i'j'}}
        \pqty*{\va{r}_{ik'}\delta_{ij'}+\va{r}_{ij'}\delta_{ik'}}+\nonumber\\
        &\omega_C\pqty*{r_{j'k'},r_{i'k'},r_{i'j'}}
        \pqty*{\va{r}_{ik'}\delta_{ij'}+\va{r}_{ij'}\delta_{ik'}}
    \end{align}
    \framebreak%

    \textbf{Truncated Polynomial form of $\omega$}
    \begin{gather}
        \begin{align}
            \omega\pqty{r_{j'k'},r_{i'k'},r_{i'j'}}=&
            \sum\limits_{l=0}^{N_\omega}
            \sum\limits_{m=0}^{N_\omega}
            \sum\limits_{n=0}^{N_\omega}
            K_{lmn}r_{j'k'}^{l}r_{i'k'}^{m}r_{i'j'}^n\nonumber\\
            &\times\pqty{L_\omega-r_{j'k'}}^C
            \Theta\pqty{L_\omega-r_{j'k'}}\nonumber\\
            &\times\pqty{L_\omega-r_{i'k'}}^C
            \Theta\pqty{L_\omega-r_{i'k'}}\nonumber\\
            &\times\pqty{L_\omega-r_{i'j'}}^C
            \Theta\pqty{L_\omega-r_{i'j'}}
        \end{align}\\
        \\
        \nonumber%
        \text{Our variational parameters are:}\quad
        K_{lmnA},K_{lmnB},K_{lmnC},L_\omega%
    \end{gather}
\end{frame}

\begin{frame}[allowframebreaks]
    \frametitle{No-Cusp Conditions}
    \textbf{Remember the Local Energy:}
    \begin{equation}
        E_L\pqty{\vb{X}\pqty{\vb{R}}}=\frac{%
            \hat{H}\psi_T\pqty{\vb{X}\pqty{\vb{R}}}
        }{%
            \psi_T\pqty{\vb{X}\pqty{\vb{R}}}
        }
    \end{equation}
    This will always contain a kinetic energy term:
    \begin{equation}
        \frac{%
            \laplacian\psi_T\pqty{\vb{X}\pqty{\vb{R}}}
        }{%
            \psi_T\pqty{\vb{X}\pqty{\vb{R}}}
        }
    \end{equation}
    Which we always want to be defined \textit{i.e.} $\psi_T$ must be twice differentiable. We demand this is also the case for the Slater determinant incorporating the backflow transformation.\framebreak%

    Consider taking our configuration of $N$ electrons, and fixing the coordinates of all but two of them. We now have a problem which can be expressed in terms of a centre of mass and separation coordinate system:
    \begin{equation*}
        \pqty{\bar{\vb{r}}_{ij},\vb{r}_{ij}}
    \end{equation*}
    Our question becomes: What constraints ought to apply to $\vb{X}\pqty{\vb{R}}$ at the coalescence point such that the local energy, $E_L$, doesn't diverge.\framebreak%

    Skipping over the derivation for brevity:\newline
    We derive the following conditions on the three-body backflow parameters ($K_{lmn}$):
    \begin{align}
        \sum_{\substack{m,n\\m+n=\alpha}}K_{1mn,s}L_\omega-K_{0mn}C & =0\quad\quad\forall\alpha\in\qty{0,\ldots,2N_\omega} \nonumber \\
        \sum_{\substack{l,n\\l+n=\alpha}}K_{l1n,s}L_\omega-K_{l0n}C & =0 \nonumber \\
        \sum_{\substack{l,m\\l+m=\alpha}}K_{lm1,s}L_\omega-K_{lm0}C & =0.
    \end{align}
\end{frame}

\begin{frame}[allowframebreaks]
    \frametitle{Analytic Derivatives}
    Numerical derivatives are unreliable, so CASINO implements analytic derivatives of its wavefunctions. This has taken up a large portion of the amount of time I've spent on this project.
\end{frame}


\begin{frame}[allowframebreaks]
    \frametitle{Further Reading}
    \bibliographystyle{vancouver}
    \bibliography{refs}

\end{frame}

\end{document}
