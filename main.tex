\documentclass[10pt]{beamer}

\usepackage{enumitem}
\usepackage{graphicx}
\usepackage{amsmath,amssymb}
\usepackage{physics}
\usepackage{animate}
\usepackage[utf8]{inputenc}

\usetheme{Berlin}

\title[Wavefunction Optimisation and Backflow in Quantum Monte Carlo Simulations]{Optimisation and Backflow:}
\subtitle{A personal experience and an introduction to my research}
\author[Clio Johnson]{Clio~Kennedy~Johnson}
%\institute[FST]{Faculty of Science and Technology}
\date[2022-12-09]{9th of December, 2022}
\titlegraphic{
    \includegraphics[scale=.014]{./images/pde-solving-silhouette.jpg}
}

\begin{document}

\begin{frame}
    \titlepage
\end{frame}


\begin{frame}
    \frametitle{Introduction \& Overview}
    \begin{itemize}
        \item[\textbullet] Overview on quantum Monte Carlo (QMC) methods, particularly variational Monte Carlo (VMC).
        \item[\textbullet] A look at wavefunction types and optimisation
        \item[\textbullet] A brief exploration of backflow displacements
    \end{itemize}
\end{frame}


\begin{frame}[allowframebreaks]
    \frametitle{Quantum Monte Carlo (QMC)}
    \textbf{First of all, what is a Monte Carlo method?}\medskip\newline
    Monte Carlo are numerical methods that obtain a result through random sampling. They involve an overall picture generated from a large number ($\mathcal{N}$) of iterations.\medskip\newline
    They are useful for systems with many, \textit{many} degrees of freedom as the error in a result scales $\propto\frac{1}{\sqrt{N}}$ irrespective of the dimensionality of a system.\medskip\newline % Mention that this is a consequence of the Central Limit Theorem.
    You can do Monte Carlo integration on integrals that looks like this:
    \begin{equation}
        E = \int\dd{\vb{x}}\rho\pqty{\vb{x}}f\pqty{\vb{x}}
        \approx\sum_{n=1}^{\mathcal{N}}\frac{f\pqty{\vb{x}_n}}{\mathcal{N}}
    \end{equation}
    \framebreak

    \textbf{A simple Monte Carlo Example:}\medskip\newline
    Compute $\pi$ from evaluating a circular integral:
    \begin{figure}
        \centering
        % Put a graphic of quarter circle MC integration here
        \animategraphics[controls,scale=0.25]{3}{./figs/pi_gif/pi_30K-}{0}{9}
        \caption{
            By nicoguaro - Own work, CC BY 3.0,
            \url{https://commons.wikimedia.org/w/index.php?curid=14609430}
        }
    \end{figure}
    \framebreak

    \textbf{Some useful terminology:}\medskip\newline
    \begin{itemize}
        \item[\textbullet] \textbf{Configuration} - A vector representing the current state of the system, in 3D QMC this is a $3N_e$ dimensional vector.
        \item[\textbullet] \textbf{}
    \end{itemize}

    \textbf{What are QMC methods?}\medskip\newline
    QMC methods aim to solve many-body Schr\"odinger equations iteratively. They are used to estimate ground state energies of these many-body systems as well as optimise wavefunctions.\medskip\newline
    \framebreak

    \textbf{Types of QMC}\medskip\newline
    There are two QMC algorithms that are most commonly used:
    \begin{itemize}
        \item[\textbullet] \textbf{Variational quantum Monte Carlo (VMC)}\newline
        Takes a trial wavefunction with electron positions, proposes a move in configuration space and then either accepts or rejects it according to the Metropolis method.\newline
        Other free parameters in the wavefunction can be optimised throughout the simulation. The simulation produces a variational estimate of the ground state energy through a rolling mean of local energies calculated at each iteration. The accuracy of this energy is dependent on a number of factors, including how well optimised the wavefunction parameters are.\newline
    \item[\textbullet] \textbf{Diffusion quantum Monte Carlo (DMC)}
    \end{itemize}
    \framebreak

    \textbf{VMC Formulation}\medskip\newline
    Given a configuration of $N$ electrons denoted
    $\vb{R}=\pqty{\vb{r}_1,\vb{r}_2,\ldots,\vb{r}_N}$, with an $N$
    body Hamiltonian $\hat{H}$.\newline
    If we provide an approximate trial wavefunction $\psi_T\pqty{\vb{R}}$. The
    variational principle ensures:
    \begin{equation}
        \int\dd{\vb{R}}\psi_T^*\pqty{\vb{R}}\hat{H}\psi_T\pqty{\vb{R}}\geq E_0
    \end{equation}
    So we're free to see the configuration $\vb{R}$ as a set of variational
    parameters, used to optimise our trial wavefunction.
    \framebreak

    \textbf{VMC Formulation Continued}\medskip\newline
    A Markov chain Monte Carlo method called the Metropolis algorithm is used
    to sample a series of points from the probability density function
    $\vqty{\psi_T\pqty{\vb{R}}}^2$.\medskip\newline
    \begin{equation}
        \int\dd{\vb{R}}
        \vqty{\psi_T\pqty{\vb{R}}}^2
        E_L\pqty{\vb{R}}
        \geq E_0,
    \end{equation}
    where
    \begin{equation}
        E_L\pqty{\vb{R}}=\frac{\hat{H}\psi_T\pqty{\vb{R}}}{\psi_T\pqty{\vb{R}}}.
    \end{equation}
    \framebreak

    \begin{figure}
        \centering
        \includegraphics[scale=0.37]{./images/control_process_diamond_ae2.png}
        \caption{Recovered autocorrelation function from the mean of 40,000
        VMC simulations on 8 cell all-electron diamond.}
    \end{figure}
    \framebreak

    \begin{itemize}
        \item[\textbullet] \textbf{Diffusion quantum Monte Carlo (DMC)}\newline
        DMC propagates walkers through imaginary time, using a birth-death
        algorithm dictated by the trial wavefunction $\psi_T$, and evaluating
        the ground state energy based on regions of walker density.
        \item Note! It relies on the fixed node approximation.
    \end{itemize}
    \framebreak

    \textbf{Fixed Nodes!}
    \begin{itemize}
        \item[\textbullet] DMC uses regions of positive walker density.
        \item[\textbullet] Nodes are points at which $\psi_T\pqty{\vb{R}}=0$.
        \item[\textbullet] Remember our system is $3N$ Dimensional.
        \item[\textbullet] Nodal structure of the wavefunction is a $3N-1$
        dimensional hypersurface.
    \end{itemize}
    \framebreak

    \begin{figure}
        \includegraphics[scale=0.22]{./images/nodal-surface-slice.png}
        \caption{
            A "slice" of the nodal surface of a 161 electron wavefunction,
            courtesy of Foulkes \textit{et al.}\cite{Foulkes2001} and Ceperley \textit{et al.}.
        }
    \end{figure}
    \framebreak

    \begin{itemize}
        \item[\textbullet] We sometimes need to alter the nodal surface of our
        trial wavefunction!
        \item[\textbullet] This can be done through VMC wavefunction
            optimisation by implementing \textbf{backflow}.
    \end{itemize}

\end{frame}

\begin{frame}[allowframebreaks]
    \frametitle{Slater-Jastrow-backflow Wavefunctions}
    \begin{equation}
        \psi_T\pqty{\vb{R}}=\mathcal{D}\pqty{\vb{R}}\exp\pqty{J\pqty{\vb{R}}}
    \end{equation}
    Normally a trial wavefunction is of the Slater-Jastrow form.\newline
    $\mathcal{D}\pqty{\vb{R}}$ is a Slater determinant.\newline
    $J\pqty{\vb{R}}$ is a Jastrow factor.\newline
    \framebreak

    \textbf{Backflow Displacement}
    \begin{itemize}
        \item[\textbullet] Nodal surface is determined by
        $\mathcal{D}\pqty{\vb{R}}$.
        \item[\textbullet] We can simulate interaction by replacing $\vb{R}$
        with a bunch of pseudo-coordinates:
        \begin{equation}
            \vb{X}\pqty{\vb{R}}=\vb{R}+\vb{\xi}\pqty{\vb{R}}
        \end{equation}
        \item[\textbullet] We give $\vb{\xi}\pqty{\vb{R}}$ an appropriate form
        dependent on several variational parameters. Allowing us to optimise the
        nodal surface variationally\cite{LopezRios2006}.
        \item[\textbullet] Leading to a wavefunction of the following form:
        \begin{equation}
            \psi_T=\mathcal{D}\pqty{\vb{X}\pqty{\vb{R}}}\exp\pqty{J\pqty{R}}
        \end{equation}
    \end{itemize}
    \framebreak

    \textbf{How to calculate Backflow displacement}
    \begin{itemize}
        \item[\textbullet] Our approximation is comprised of two-body,
        electron-nucleus, and electron-electron-nucleus terms.
        \item[\textbullet] I am currently attempting to implement a three-body
        backflow contribution (electron-electron-electron interactions).
    \end{itemize}
    \framebreak

    \begin{align}
        \vb{\xi}_i\pqty{\vb{R}}=&
        \vb{\xi}_i^{\text{e-e}}\pqty{\vb{R}}+
        \vb{\xi}_i^{\text{e-n}}\pqty{\vb{R}}+
        \vb{\xi}_i^{\text{e-e-n}}\pqty{\vb{R}}+
        \vb{\xi}_i^{\text{e-e-e}}\pqty{\vb{R}}
        \\
        \vb{\xi}_i^{\text{e-e-e}}\pqty{\vb{R}}=&
        \sum\limits_{\substack{j\neq i\\j=1}}^{N}
        \sum\limits_{\substack{k\neq i\\k>j\\k=1}}^{N}
        \omega_A\pqty{r_{j'k'},r_{i'k'},r_{i'j'}}
        \pqty{\vb{r}_{ik'}\delta_{ij'}+\vb{r}_{ij'}\delta_{ik'}}+\nonumber\\
        &\omega_B\pqty{r_{j'k'},r_{i'k'},r_{i'j'}}
        \pqty{\vb{r}_{ik'}\delta_{ij'}+\vb{r}_{ij'}\delta_{ik'}}+\nonumber\\
        &\omega_C\pqty{r_{j'k'},r_{i'k'},r_{i'j'}}
        \pqty{\vb{r}_{ik'}\delta_{ij'}+\vb{r}_{ij'}\delta_{ik'}}
    \end{align}
    \framebreak

    \textbf{Truncated Polynomial form of $\omega$}
    \begin{gather}
        \begin{align}
            \omega\pqty{r_{j'k'},r_{i'k'},r_{i'j'}}=&
            \sum\limits_{l=0}^{N_\omega}
            \sum\limits_{m=0}^{N_\omega}
            \sum\limits_{n=0}^{N_\omega}
            c_{lmn}r_{j'k'}^lr_{i'k'}^mr_{i'j'}^n\nonumber\\
            &\times\pqty{L_\omega-r_{j'k'}}^C
            \Theta\pqty{L_\omega-r_{j'k'}}\nonumber\\
            &\times\pqty{L_\omega-r_{i'k'}}^C
            \Theta\pqty{L_\omega-r_{i'k'}}\nonumber\\
            &\times\pqty{L_\omega-r_{i'j'}}^C
            \Theta\pqty{L_\omega-r_{i'j'}}
        \end{align}\\
        \\
        \nonumber
        \text{Our variational parameters are:}\quad
        c_{lmnA},c_{lmnB},c_{lmnC},L_\omega
    \end{gather}
\end{frame}

\begin{frame}[allowframebreaks]
    \frametitle{No-Cusp Conditions}
    \textbf{Remember the Local Energy:}
    \begin{equation}
        E_L\pqty{\vb{X}\pqty{\vb{R}}}=\frac{
            \hat{H}\psi_T\pqty{\vb{X}\pqty{\vb{R}}}
        }{
            \psi_T\pqty{\vb{X}\pqty{\vb{R}}}
        }
    \end{equation}
    This will always contain a kinetic energy term:
    \begin{equation}
        \frac{
            \laplacian\psi_T\pqty{\vb{X}\pqty{\vb{R}}}
        }{
            \psi_T\pqty{\vb{X}\pqty{\vb{R}}}
        }
    \end{equation}
    Which we always want to be defined \textit{i.e.} $\psi_T$ must be twice
    differentiable. We demand this is also the case for the Slater determinant
    incorporating the backflow transformation.
    \framebreak

    Consider taking our configuration of $N$ electrons, and fixing the
    coordinates of all but two of them. We now have a problem which can be
    expressed in terms of a centre of mass and separation coordinate system:
    \begin{equation*}
        \pqty{\bar{\vb{r}}_{ij},\vb{r}_{ij}}
    \end{equation*}
    Our question becomes: What constraints ought to apply to
    $\vb{X}\pqty{\vb{R}}$ at the coalescence point such that the local energy,
    $E_L$, doesn't diverge.
    \framebreak

    Skipping over the worst parts of the derivation for now:\newline
    Recalling the kinetic contribution to local energy, we want to consider
    the limit as $r_{ij}$ goes to $0$:
    \begin{equation}
        \lim\limits_{r_{ij}\to 0}\frac{
            \laplacian\psi_T\pqty{\vb{X}\pqty{\vb{r}_{ij}}}
        }{
            \psi_T\pqty{\vb{X}\pqty{\vb{r}_{ij}}}
        }
    \end{equation}
    If the two electrons are indistinguishable (spin aligned), then the
    denominator must equal zero in the limit.\medskip\newline
    So to prevent problems at like-spin electron coalescence we demand
    \begin{equation}
        \lim\limits_{r_{ij}\to 0}\laplacian\psi_T\pqty{\vb{X}\pqty{\vb{R}}}=0
    \end{equation}
    \framebreak

    This is what allows us to derive conditions on the $c_{lmn}$ variational
    parameters. These are still in development, we've reached the limit of my
    research so far.\medskip\newline
    Here be dragons...
\end{frame}


\section{References}
\begin{frame}[allowframebreaks]
    \frametitle{Further Reading}
    \bibliographystyle{vancouver}
    \bibliography{refs}

\end{frame}

\end{document}
